\documentclass[12pt,a4paper]{article}

\usepackage[utf8]{inputenc}
\usepackage[english]{babel}
\usepackage[english]{isodate}
\usepackage[parfill]{parskip}



\begin{document}

\pagenumbering{Roman}
\tableofcontents
\newpage
\listoffigures
\newpage
\listoftables
\newpage
\pagenumbering{arabic}

\section{Introduction}
The goal of this file is to describe the steps to create an eletric bicycle. I dream of a world where we can all run on renewable energy sources. Building my own project in this field just me make inserted in this new trend. I would you enjoy.

\section{Main components}

\begin{enumerate}
\item Bicycle
\item Motor
\item Battery
\item Recharger

\end{enumerate}

\section{Tools}


\section{Batteries}

In order to better understand batteries, one should focus on its main parameters that define its quality:



\begin{enumerate}
\item \textit{Power-to-ratio} %%https://en.wikipedia.org/wiki/Power-to-weight_ratio
\item \textit{Specific energy}. It corresponds to the amount of energy per unit mass.
\item \textit{Energy density}. It amounts for the quantity of energy in a given volume.
\end{enumerate}
The two main options of batteries are: lithium and lead acid battery. Our book will describe firstly the lithium batteries.

\section{Lithium}

Some types of batteries are:

\begin{enumerate}
\item \textit{Lithium Iron Phosphate} $LiFePO_4$. Also called LFP battery, they are some of the heaviest and most expensive ones, but they are also one of the safest and longlasting ones. They have low cost, are non toxic, safe and has a lot of thermal stability. 
\item \textit{Lithium Manganese Oxide} % https://en.wikipedia.org/wiki/Lithium_ion_manganese_oxide_battery
\end{enumerate}
\end{document}
% http://www.ebikeschool.com/electric-bicycle-batteries-lithium-vs-lead-acid-batteries/

\subsection{Location}


